%\documentclass[12pt]{article}

%\usepackage{../Diss}
%\doublespacing

%\setcounter{section}{-1}

%\begin{document}

 %\author{}
 %\title{Chapter 1\\ \ \\ Remarks on Sense Data}
 %\date{}
%\maketitle 

\section{Introduction}
The two parts of this chapter each defend sense-data against a different argument. The first section defends against part of C.D. Broad's \citeyear{Broad:1952} argument against \em direct apprehension\em\ accounts of perception. The second section takes on the charge that sense-data must in some sense disobey the Law of Excluded Middle. 

Each of the arguments has the form of a reductio. Apart from questioning a premise, a defense of a philosophical thesis against a reductio has at least approaches available. One could argue that the alleged absurdum is not in fact absurd, perhaps on the grounds that it does not entail a logical contradiction. \S\ref{broad} takes this tack in examining C. D. Broad's \citeyear{Broad:1952} argument against sense-data accounts of perception. I propose a way to block the argument by accepting the putative absurdum that it is logically possible that one may apprehend another's dreaming sensations. 

Another, more drastic, defense against a reductio embraces the putative absurdum and accounts for its tenability by way of logical revision. \S\ref{sdandlem} takes this strategy in response to W. H. F. Barnes's \citeyear{Barnes:1944} argument that sense data, if they do exist, must disobey the Law of Excluded Middle. I propose a minimal revision to classical logic that allows for such strange behavior for atomic objects, while blocking the inference to a contradiction (in this context, the failure of the Law of Non-contradiction).

Proposing a revision of logic to block a conclusion in many ways seems a doomsday response which could easily be avoided by simply rejecting the problematic theory it is intended to save. This may seem especially distasteful in the service of an out-of-favor theory of perception like sense-data. For even if Barnes's argument is blocked, one might claim, there are many other arguments more well-rehearsed that sense-data must overcome. And to be sure, a full defense of sense-data is not forthcoming.

The case of sense-data, however, bears an interesting parallel to better known arguments for logical revision. The truth-theoretic paradoxes, for instance, have driven many to propose logical revision as a result of the transparency of the truth predicate:

\[\True{p} \iff p\]

\noindent Traditional realist theories that give an `act-object' analysis of perception appeal to a relation of direct acquaintance or apprehension. In sense-data, or \emph{indirect} realist, theories, this relation of direct apprehension provides a transparency principle for appearance, similar to that for truth. The nature of perception, according to the indirect realist, puts us in direct acquaintance with objects, such that we cannot be mistaken about appearances. This certainty cannot, however, be of the nature of the things we ordinarily take ourselves to perceive, for clearly the following principle of appearances is false:

\[\Appears{F\upalpha}\then F\upalpha\]

\noindent The converse is surely false, too. Indeed, \S\ref{extreme} examines Broad's refutation of a weaker version of this na\"{i}ve direct realist principle:

\[\Appears{F\upalpha}\then \exists ! \upalpha \]

Sense-data theorists, then, account for the certainty of apprehension indirectly, as consisting in a correlation between an ordinary physical object's appearing $F$ and something \emph{or other} being $F$. This something or other is the immediate object of perception and it bears some close relation with the physical object we take ourselves to perceive. At minimum this suggests something akin to the following principle:

\[\Appears{F\upalpha}\then \exists y Fy \]

\noindent Thus something appearing $F$ implies, not that \emph{it} is $F$, but that some immediate object of perception is $F$. Further, there is a strong intuition that, given the role sense-data are to play in a theory of perception, not only can we not be mistaken about their positive features, but they also cannot have any positive phenomenological features that we do not directly apprehend. Some accounts of sense-data, then, accept the bi-conditional

\[\Appears{F\upalpha}\iff \exists y Fy \]

\noindent for some suitable restriction on $F$ that only considers phenomenological properties. \S\ref{sdandlem} examines these latter two principles and the potential resulting paradoxes in light of Barnes's argument that the first, weaker, principle entails that sense-data violate the Law of Excluded Middle. 

The result, if not an enduring defense of sense-data, is a connection between paradoxes resulting from \emph{semantic} theories and those resulting from \emph{metaphysical} theories. Later chapters examine a similar parallel that might be made with paradoxes from \emph{scientific} theories. The broader approach, then, is not necessarily to defend any particular theory, but to attempt to understand what features these disparate motivations for logical revision might share, and where they may differ.  

\section{Broad's argument against sense data}\label{broad}
Broad argues against prehensive accounts of the epistemological character of perception. He has two arguments from dreams and hallucinations. The first refutes the view that sense-perception consists in the prehension of physical objects. The second argues against the view that sense-perception consists in the prehension of sense-data. 

There are two ways one might speak of the `character' of perception. The \em epistemological\em\ character accounts for the way in which our perception puts us in a relation with ordinary physical objects, such that we are able to gain knowledge of those objects. It describes the cognitive process of perception which puts us ``in touch'' with physical objects. 

Just as the epistemological character of perception is the way in which perception \em does\em\ connect us with physical objects, the \em phenomenological character\em\ is the way in which perception \em seems\em\ to so connect us with objects. That is, the phenomenological character is how the epistemological character is presented to us by experience, such that \em if\em\ the phenomenological character of a given form of sense perception were accurate, it would be \em identical\em\ to its epistemological character.

The question then becomes, \em Is\em\ the common-sense description of phenomenological character accurate? Is the phenomenological character of vision a good guide to its epistemological character? Broad attacks two possible answers to these questions. The first is the \em extreme view\em, which holds that phenomenological character is a \em perfect\em\ guide to epistemological character. Visual experience presents itself as the direct apprehension of objects, so, the extreme view says, we must in fact directly apprehend objects. The second is the \em moderate view\em, which holds that, although phenomenological character is not a \em perfect\em\ guide to epistemological character, it is still \em some\em\ sort of a guide. We may be wrong to infer from phenomenology that we are directly apprehending ordinary physical objects, but we are correct to infer that we are directly apprehending \em something or other\em.

\subsection{Broad's argument against the extreme view}\label{extreme}
The phenomenology of sense-perception seems to inform us that, when we look at objects, we are directly apprehending (or \em prehending\em) them. Physical objects \emph{leap} out of the place they appear, such that we are directly aware of them. Since the extreme view holds that phenomenological character is a complete guide to epistemological character, the theory states that visual perception is the direct prehension of objects. This view is the target of Broad's first argument. Whatever else it may hold, such a theory is committed to the following:
\begin{enumerate}

\item[(\PRO)] A form of sense-perception $\upvarphi$ (e.g. see, hear, or touch) has the epistemological character of \em Prehension of Objects\em\ only if an external object's being at a certain place $p$ at a certain time $t$ is a \em necessary condition\em\ for a person's having an experience which he would naturally describe as $\upvarphi$-ing an external object at $p$ at $t$. 

\end{enumerate}
If visually perceiving is the direct prehension of ordinary physical objects, then in order to be in such a prehension relation with an object at $p$ at $t$, there must be an object there and then with which to be in the relation.

There is a certain plausibility to this account with ordinary cases of veridical perception. When I veridically perceive a burrito in front of me at arms length, it seems that I am directly and immediately apprehending\textemdash``grasping'' in some sense\textemdash the burrito. In this case the necessary condition stated in \PRO\ is satisfied.

Given, though, that dreams and hallucinations are phenomenologically indistinguishable from waking sense-perception, there is a problem for this account.\footnote{Whether dreams are \em phenomenologically\em\ indistinguishable from  waking sense perception I will not debate here. Surely, just because I cannot \em know\em\ whether I am dreaming or waking right now, it does not follow that dreams and waking perception are \em phenomenologically\em\ indistinguishable, for this is perhaps merely an epistemological concern. Broad points out, however, that just because we do distinguish between waking and dreaming we do not do so ``by noting dissimilarities in their phenomenological character. We do so by considering the interrelations of experiences with the earlier and later experiences of the same person and the contemporary experiences of others'' \cite[p. 41]{Broad:1952}. This is not clearly true, and it is hard to imagine what evidence would count one way or the other, since it would require a suitable criterion for phenomenological indistinguishability. But it does seem plausible to assert that dreams definitely \em seem\em\ different.} Broad writes:

\begin{quote}
There are certain experiences, viz., dreams and waking hallucinations, which exactly resemble normal waking sense-perceptions in all their phenomenological characteristics (including that of being ostensibly prehensive of foreign bodies and external physical events), but which are certainly not in fact prehensions of any such objects. It seems most unlikely that experiences which exactly resemble these in all their phenomenological characteristics, as do normal waking sense perceptions, should be fundamentally unlike them in their epistemological character.

...On the whole, then, I see nothing for it but to draw the following conclusion. Our waking experiences of seeing, hearing, and touching are not, as they appear to us to be, prehensions of foreign bodies and physical events and of certain of their intrinsic qualities.\cite[p. 41]{Broad:1952}
\end{quote}
Broad's argument against \PRO\ from dreams and hallucinations seems to be the following:

\begin{enumerate}

\item Dreams and waking hallucinations have the same phenomenological character as normal waking sense-perceptions. [Premise]
\item If dreams/hallucinations and normal waking perceptions have the same phenomenological character, then they have the same epistemological character. [Premise]
\item Thus, dreams and waking hallucinations have the same epistemological character as normal waking sense-perceptions. [From (1) and (2)]
\begin{enumerate}
\item[4.] Suppose that waking sense-perception has the epistemological character \PRO. [Reductio premise]
\item[5.] Thus, dreams and waking hallucinations must have the epistemological character \PRO. [From (3) and (4)]
\item[6.] However, one can have an experience during a dream that one would naturally describe as seeing an external object at some place and time, while there is no external object there and then. [Premise]
\item[7.] Hence, the epistemological character of dreams is not \PRO. This contradicts (5). [From (5) and (6)]
\end{enumerate}
\item[8.] Therefore, our supposition (4) is false. That is, waking sense-perception does not have the epistemological character \PRO. [From (4)-(7)]
\end{enumerate}

Premise (1) is from the traditional assumption that dreams and hallucinations are phenomenologically indistinguishable from waking sense-perception. Premise (2) is an apparent commitment of the extreme view. If the extreme view infers epistemological character directly from phenomenological character, then it seems one must be willing to apply the inference for all forms of perception, waking or dreaming. (We will return to this in the next section.) This leads us to contradiction: in dreaming perception we seem to perceive objects when no objects are present (6), thus failing to meet the necessary condition in \PRO.

\subsection{Broad's argument against the moderate view}\label{moderate}

It seems we listened to our phenomenology too much: we do not directly prehend physical objects. It does seem, however, that we are directly prehending \em something\em. If it is not the ordinary objects of perception, there must be some other kind of thing that we prehend, in virtue of which we perceive ordinary objects. Since these sense data are not `ordinary' objects, they must be something else. The moderate view, then, holds that, although phenomenology may not tell us anything about what \emph{kinds} of objects we are directly aware of, we can still infer \em something\em\ about epistemological character from phenomenological character\textemdash namely, that the \em relation\em\ we are in is one of prehension. 

Broad's argues against the moderate view by examining what follows merely from the existence of a prehension relation between a subject and a particular. As before, we need only focus on a necessary condition for the moderate view:
\begin{enumerate}

	\item[(\PRS)] A form of sense-perception $\upvarphi$ has the epistemological character of \em Prehension of Sensa\em\ only if a person $S$ $\upvarphi$-ing consists in (or involves) a relation $R$\textemdash the target relation here being prehension\textemdash such that:
	\begin{enumerate}
		\item There is some particular, $x$ such that $\langle S,x\rangle \in R$.
		\item If $\langle S,x\rangle \in R$, then it is \em logically possible\em\ that for some $S^*$ such that $S\neq S^*$, $\langle S^*,x\rangle \in R$.
	\end{enumerate}

\end{enumerate}
Broad outlines other constraints on $R$ which fall out of its being a prehension relation, but it is only (b) that will concern this next argument. Note that Broad is ultimately arguing that the epistemological character of sense-perception does not consist in a prehension relation of \em any\em\ kind. The first argument shows that the epistemological character of sense-perception is not that of \PRO. If the only other available option is \PRS, and Broad successfully refutes \em that\em, then he has shown that sense-perception does not consist in a subject being in a prehension relation with a particular.

The argument against \PRS\ follows a similar structure as before.
\begin{enumerate}

\item[9.]Dreams and hallucinations have the same phenomenological character as waking sense-perceptions. [Premise]
\item[10.]If dreams/hallucinations and normal waking perceptions have the same phenomenological character, then they have the same epistemological character. [Premise]
\item[11.]Thus, dreams and waking hallucinations have the same epistemological character as normal waking sense-perceptions. [From (9) and (10)]
\begin{enumerate}
\item[12.]Suppose that waking sense-perception has the epistemological character \PRS. [Reductio premise]
\item[13.]Thus, dreams and waking hallucinations must have the epistemological character \PRS. [From (11) and (12)]
\item[14.]However, in the case of $S$ dream-seeing, it is not logically possible that some $S^*$ (such that $S\neq S^*$) prehend the alleged particular of $S$'s dream. [Premise]
\item[15.]Hence, the epistemological character of dreams is not \PRS. This contradicts (13). [From (13) and (14)]
\end{enumerate}
\item[16.]Therefore, our supposition (12) is false, and so waking sense-perception does not have the epistemological character \PRS. [From (12)-(15)]

\end{enumerate}
The form of this argument is similar to the previous one. Here I suggest that the sense-data theorist can deny (14), and accept the logical possibility of another person experiencing my dream sensing. 

Broad takes this as absurd, but we should look more closely at the reasons behind accepting premises (2) and (10). As the premises are stated they are identical, but there are important differences in their respective justifications. Premise (2) seems to be an application of a more general principle:
\begin{enumerate}

\item[(EP)] For all forms of sense-perception $\upvarphi_1$ and $\upvarphi_2$, if $\upvarphi_1$ and $\upvarphi_2$ have the same phenomenological character, then $\upvarphi_1$ and $\upvarphi_2$ have the same epistemological character.

\end{enumerate}
One may wonder why we should accept this. After all, Broad is claiming that the phenomenological character of sense-perception (viz. sight) is \em not\em\ a good guide to its epistemological character. If this is so, why think that two forms of sense-perception (or two modes of one form) which share phenomenological character will share epistemological character? If length is not a good guide to its width, then there is no reason to think that two things of the same length will have the same width. 

Since Broad recognizes that (EP) is false, in order for his argument to go through, he will have to commit his opponents to it. He might be able to do so for the proponents of (\PRO).\footnote{Note, he may not need do this for \em refuting\em\ (\PRO). He has a simpler argument against (\PRO): we see objects in mirrors when none is present, and we see stars that are not there. Here we are concerned only with defending \ the argument from dreams and hallucinations.} In the case of sight, the reason for thinking it consists in (or at least involves) the direct prehension of physical objects is simply that its phenomenological character seems to do just so. If this is the \em only\em\ phenomenological reason to think sight obeys (\PRO), then it would seem to commit the proponents of (\PRO) to (EP) and thus to (2).

But the commitments of proponents of (\PRS) are not as straightforward. Note that an even more general principle Broad is challenging is:
\begin{enumerate}

\item[(G)]For all forms of sense-perception $\upvarphi$ and all characters $\updelta$, if $\upvarphi$ has phenomenological character $\updelta$, then $\upvarphi$ has epistemological character $\updelta$.

\end{enumerate}
This implies (EP). However, as Broad points out, proponents of (\PRS) (i.e. sense-data theorists) do not wholly accept (G). He acknowledges that, to the extent that philosophers saw the inadequacy of (\PRO), ``they felt obliged to hold that the phenomenological character of [sense-experiences] is a misleading guide to their epistemological character'' (42). It is clear, then, that proponents of (\PRS) do not accept (G) outright. So, the sense-data theorists reject that the epistemological character of a given form of sense-perception is exactly like its phenomenological character. It is left to explain, however, whence comes the commitment to sense-data.

Though phenomenological character may not be a \em perfect guide\em\ to epistemological character, the moderate view maintains that it is at least \em some\em\ such guide. Phenomenology may deceive us as to exactly what kind of objects we are prehending, but it does inform us \em that\em\ we are prehending something or other. It seems the sense-data theorist accepts, then, that if the phenomenological character of sight is that of prehension of \em something\em, then the epistemological character of sight is that of prehension of \em something\em. We see here the commitment to the following principle:
\begin{enumerate}

\item[(SR)] For all $\upvarphi$, if $\upvarphi$ has the phenomenological character of a relation $R$ to some kind of object $X$, then $\upvarphi$ has the epistemological character of relation $R$ to some kind of object $Y$.

\end{enumerate}
Note that, importantly, $X$ need not be identical to $Y$. In fact, as the argument against (\PRO) shows, $X$ \em cannot\em\ be the same kind of thing as $Y$.

The sense-data theorist thus accepts (10) insofar as it follows from (SR). Perhaps, then, the following is a better formulation:
\begin{enumerate}

\item[10$'$.] If the phenomenological character of dreams/hallucinations and waking sense-perception both consist in relation $R$, then the epistemological character of dreams/hallucinations and waking sense-perception both consist in relation $R$.

\end{enumerate}
The important point here is this: Just as the sense-data theorist maintains that \em seeming\em\ to prehend a certain kind of object (i.e. a physical object) is no guide to what kind of objects we \em actually\em\ prehend (i.e. sense-data), he can also consistently maintain that the kind of objects we actually prehend in \em waking\em\ sense-perception is no guide to what kind of objects we prehend in \em dream\em\ sense-perception. 

This returns us to constraint (b) in (\PRS). Russell, Moore, and here Broad, are careful to emphasize this constraint on waking sense-data. But one wonders, Why think it is logically possible, in the case of waking sense-data, that someone else could have prehended the very same sense-data that I prehend? One fairly straightforward answer is that the relation of prehension is a relation between a subject and a non-identical particular, and for any such relation, it is logically possible that each of the relata be in the same relation with a distinct relatum. 

This suggests one of several reasons for the possibility of prehending another's waking sense-data. Moore, for instance, writes:

\begin{quote}
I think, then, that the term `sensation' is liable to be misleading, because it may be used in two different senses, which it is very important to distinguish from one another. It may be used \em either\em\ for the colour which I saw or for the experience which consisted in my seeing it. And it is, I think very important, for several reasons, to distinguish these two things. In the first place, it is, I think, quite conceivable (I do not say it is actually true) but \em conceivable\em\ that the patch of colour which I saw may have continued to exist after I saw it: whereas, of course, when I ceased to see it, \em my seeing\em\ of it ceased to exist \citeyear[p. 31]{Moore:1953}.
\end{quote}
Russell distinguishes between sensibilia and sense-data, the latter being sensibilia that are apprehended. He writes:

\begin{quote}
We cannot ask, `Can sense-data exist without being given?' for that is like asking, `Can husbands exist without being married?'\dots Unless we have the word \em sensibile\em\ as well as the word `sense-datum', such questions are apt to entangle us in trivial logical puzzles.

It will be seen that all sense-data are \em sensibilia\em. It is a metaphysical question whether all \em sensibilia\em\ are sense-data, and an epistemological question whether there exist means of inferring \em sensibilia\em\ which are not data from those that are \cite[pp. 110-11]{Russell:1917}.
\end{quote}
Perhaps we should treat this constraint as sufficient for logical possibility as required by (b):
\begin{enumerate}

\item[(L)]For all $x$ and $y$, $xRy$ where $x\neq y$, only if it is logically possible that there is some $z$ such that $zRy$ where $x\neq z$.

\end{enumerate}
It is hard to know exactly what Broad thought logical possibility amounts to, so I will suggest constraints that give a fairly broad conception of logical possibility.

With the development of non-standard logics, the question is no longer clearly whether something is logically possible simpliciter, but whether it is logically possible \em according to a certain logic\em. Or, if one thinks of possibility as governed by logical \em laws\em, we must consider which logic the laws of which we should inspect. For a logical monist who thinks there is one ``true logic,'' this is tantamount to asking what the \em real\em\ logical laws are.

However, even if it is clear which logic to choose, formulating a criterion of logical possibility is not straightforward. In classical propositional logic, our resources are scarce. The obvious candidate there is to define possibility as truth in some model, and necessity as truth in all models. Here any sentence but the negation of a logical truth will be possible. Perhaps this is all that logical possibility amounts to. On this account we have limited ability to express inferences with respect to possibility in the object language. But if this is all that is meant by logical possibility, then since (14) is not straightforwardly a denial of a classical logical truth, it follows that accepting (14) brings no contradiction.

If one takes standard Kripke modal logic (e.g. \textsf{S5}) to in some sense model logical possibility, things are perhaps less clear. It may be unproblematic on this account to talk of \em valid inferences\em\ with respect to possibility, but to formulate a criterion of logical possibility requires filling in the schema:

\begin{enumerate}
\item[(S)] $p$ is logically possible iff \dots
\end{enumerate}
Note that (L) can be formulated several ways, including:

\begin{enumerate}
\item[(L$^*$)]$\forall x\forall y((xRy\wedge x\neq y)\supset \Diamond\exists z(zRy\wedge z\neq x))$
\end{enumerate}
and an inferential reading:

\begin{enumerate}
\item[(L$^{**}$)]$\exists x\exists y(xRy\wedge x\neq y)\vdash\exists x\exists y((xRy\wedge x\neq y)\wedge\Diamond\exists z(zRy\wedge x\neq z))$
\end{enumerate}
Under the assumption that \textsf{S5} or a similar logic correctly models logical possibility, there are several options available for filling in (S). We can talk of truth in \em some\em\ model, truth in \em all\em\ models, or perhaps truth in \em the correct\em\ model. The first two options seem not to give us what we want. If we adopt \textsf{S5} as our paradigm, and define logical possibility in terms of truth in some model, we get the following:
\begin{enumerate}
\item[(S$_1$)] $p$ is logically possible iff $\Diamond p$ is true at $w_0$ in some model.
\end{enumerate}
Here the only logical impossibilities are the negations of classical logical truths with a few additions, such as $\Box\neg A\wedge\Box A$. Similarly, if we opt for
\begin{enumerate}
\item[(S$_2$)] $p$ is logically possible iff $\Diamond p$ is true at $w_0$ in all models.
\end{enumerate}
the only logical possibilities are the logical necessities. Clearly, neither option lends much support to (14). The last seems to be our best candidate. Thus:

\begin{enumerate}
\item[(S$_3$)] $p$ is logically possible iff $\Diamond p$ is true at $w_0$ in the correct model of the universe.
\end{enumerate}
This is of course contentious for anyone who believes that the `models' at issue cannot completely represent the universe. But under the assumption that \textsf{S5} correctly models logical possibility, it seems the most eligible candidate for a criterion. Of course, in order to argue that a given claim is or is not logically possible, it is hard to see how one can make non-question begging assumptions about the correct model. Here our target $p$, without any further constraints, comes out contingent, i.e. (L$^*$) is true at $w_0$ in some models but not in others, and (L$^{**}$) comes out invalid, but the set consisting of the premise and conclusion is satisfiable. Plausibly, though, if we are looking for the correct model, we are not looking at validity across all models, but truth in \em the\em\ model\textemdash and thus not truth in all or merely some models. However, putting the requisite constraints on \em the\em\ model (i.e. picking out a set of candidate models) such that a form of (L) comes out true will only serve to beg the question.

But to get a hold on the dialectic, consider that Broad takes (14) to be an absurdum for the sense-data theorist. Though we have not shown the truth of (14), it is hard to see how we should be convinced to find an absurdum therein.

We can, however, rule out reasons for accepting (14) relating to considerations about $R$. One can perfectly well say that during dreams one is in a prehension relation with some particular or other for which it is \em logically possible\em\ that someone else be in a prehension relation to that same particular. To see what this is asserting, consider what it does not assert. The following two considerations are \em not\em\ reasons for asserting (14):
\begin{enumerate}

\item[(i)]Waking sense-data are causally dependent on physical objects, whereas dreaming sense-data are not.
\item[(ii)]Though waking sense-data may be existentially dependent on there being some mind or other that prehends them, they are not existentially dependent on a particular mind, as they are in the case of dreaming sense-data. 

\end{enumerate}
(i) is irrelevant to the consideration of (b). As Broad notes, the point here is not about causal or nomological possibility, but logical possibility. More importantly, given what (10$'$) asserts, there is no reason to hold that the nature of waking sense-data tells us anything whatever about the nature of dreaming sense-data. Thus, it seems that (ii)
\textemdash or any similar consideration\textemdash gives us no reason to deny that it is logically possible for one to prehend another's dreaming sense-data. 

We might be drawn to believe (14) because it must be the very \em nature\em\ of these supposed dream sense-data that they are private and they belong essentially to the particular dreamer. But we have shown that the moderate view can hold that dream sense-data are essentially private, while still holding that it is logically possible that another prehend one's dream sense-data. The sense of logical possibility at issue here is the absence of logical contradiction. It is no problem to coherently think about or model any subject prehending a particular. And this point is all that the sense data theorists like Moore and Russell need. It remains open at this point to talk about the natures of these particulars and whether their natural features preclude in any more narrow sense prehension by others. The sense-data theorist can make the requisite inferences from phenomenology without commitment to a particular conception of the nature of these particulars, nor to there being only one such nature. For the inferences drawn from phenomenology only tell us about the nature of the relation constitutive of perception, but not about the natures of the relata.




\section{Sense data and LEM}\label{sdandlem}
In this section, we are faced with another argument that seems to be a reductio against sense-data theory. The absurdum in this case is that, if sense-data do exist, they disobey the Law of Excluded Middle (herein, LEM). I will examine several things one might say in response to the argument. The argument is in several ways unclear, and I will look at several possible clarifications. However, it will seem that none of the responses conclusively block the argument. Thus, I will embrace the putative absurdum, and accept the conclusion that sense-data disobey (in some sense) LEM. 

\subsection{Barnes's argument}

W. H. F. Barnes (1944) raises several objections to sense-data theories. I would like to focus on just one particular argument, which charges sense-data (if they exist) with disobeying the LEM. Barnes writes:

\begin{quote}
If I contemplate an object at some distance, it often happens that I am uncertain whether it is circular or polygonal. It is necessary for me to approach close before I can determine the matter with certainty. On the [sense-data] theory, the mode in which the object appeared to me at first is a sensum, every sensum is what it appears to be. Now this sensum appears neither circular nor non-circular. Therefore it is neither circular nor non-circular. (1944: 145)
\end{quote}

\noindent Barnes's argument seems to be the following:

\begin{enumerate}[i.]
\item If a sense-datum appears F, then it is F.
\item Sense-datum $s$ appears neither circular nor non-circular.
\smallskip
\hrule width 325pt
\item Therefore, $s$ is neither circular nor non-circular.
\end{enumerate}

\noindent If this argument is successful, it seems to show that sense-data in some disobey the Law of Excluded Middle. Barnes apparently takes this to be a reductio of sense-data theories that accept some form of (i). Indeed this would commit the sense-data theorist to some peculiarly behaving entities. However, I intend to show that it need not follow from (iii) that sense-data theories are incoherent.

The first premise is the assumption of incorrigibility from the target sense-data theories. The relation of apprehension is such that I cannot be mistaken about features of my own sense-data of which I am aware. The second premise comes from Barnes's proposed example of an object that appears neither to have nor lack a certain feature.

There are few intial ways one might respond to the argument. One could (a) deny across the board the occurrence of phenomena of the type Barnes proposes, thus denying that (ii) is ever true, (b) question the inference to (iii), or (c) embrace the conclusion. Though I will ultimately go for (c), let us first consider (a) and (b).

One possible objection to make is to question the scope of `appears' in (ii). Is it the case that an object determinately \emph{appears} a certain way, where that way is neither circular nor non-circular? Or, is it rather that the object does not determinately appear to be circular nor determinately appear to be non-circular? I am not sure how to gather evidence for the truth of one reading over the other, but Barnes insists that an instance of a sense-datum appearing neither $F$ nor non-$F$, as (ii) intends to pick out, is not merely a case of an object failing to appear a certain way.

At this point, however, it is unclear how to take (ii). From the form of the argument, it seems that `neither circular nor non-circular' must be a property that $s$ positively appears to have, in order to detach (i), thus:

\begin{enumerate}
\item[] $F$: is neither circular nor non-circular
\end{enumerate}
Now, it would be fine (for the purposes of detaching (i)) if $F$ were a conjunctive property, for instance:

\begin{enumerate}
\item[] $G$: is non-circular
\item[] $H$: is non-non-circular
\item[] $F*$: is $G$ and $H$
\end{enumerate}
In which case we would restate (ii):

\begin{enumerate}
\item[ii$'$.] $s$ appears non-circular and $s$ appears non-non circular.
\end{enumerate}
However, $F*$ seems to be more like a ``contradictory'' property, and not some sort of ``gappy'' property. But if we take (ii) like the following:

\begin{enumerate}
\item[ii$''$.] $s$ does not appear circular and $s$ does not appear non-circular.
\end{enumerate}
then the argument as stated does not go through. In order for (ii$''$) to work, we would need to read (i) as:

\begin{enumerate}
\item[i$'$.] A sensum appears F iff it is F.
\end{enumerate}
This would amount to claiming that sense-data are all and \em only\em\ what they appear to be. Broad \citeyear{Broad:1927} and others argue against this principle, though, and it does not seem that Barnes is assuming it. As the argument is stated, it is unclear how to take (ii) with respect to $F$. (ii$'$) does not seem to get us (iii), but instead to something like:

\begin{enumerate}
\item[iii$'$.] $s$ is both non-circular and non-non-circular.
\end{enumerate}
And (ii$''$) requires (i$'$) and not merely (i) in order to get to (iii). 

One might conclude that this argument trades on some confusion. But if so, it is difficult to precisely locate. Instead one might accept (iii) and take it that it expresses a failure of sense-data to obey LEM. 

Although Barnes's argument serves as the initial motivation for the logic, there are other possible applications as well. With respect to sense-data, one might treat such phenomena as the waterfall illusion with truth-value gaps. Further, Putnam \citeyear{Putnam:1957} noted that quantum physics might require a gappy logic that rejects LEM. The target logic would allow for such a model that keeps the gaps at the quantum level. We will discuss many of these issues in subsequent chapters.

\subsection{LEM and LNC}
There are several ways to formulate LEM and LNC precisely. Broadly speaking, LEM states that every sentence\footnote{I use the term `sentence' here and throughout to refer to a meaningful, declarative sentence.} is either true or false, and LNC states that no sentence is both true and false. One might, for instance, distinguish between \em not accepting\em\ LEM and \em denying\em\ it. Formally, this might involve introducing two types of negation. Or one might formulate the laws using modal operators. For our purposes, however, it will suffice to forgo such distinctions, and formulate the two laws in strong form:

\begin{quote}
\begin{enumerate}
\item[(LEM)]$B\Vdash A\vee\neg A$
\item[(LNC)]$A,\neg A\Vdash B$
\item[(LNC*)]$B\Vdash\neg(A\wedge\neg A)$
\end{enumerate}
\end{quote}

Note that in classical propositional logic, the following inference is valid:

\begin{center}
$\neg(A\vee\neg A)\Vdash A\wedge\neg A$
\end{center}

\noindent Thus, to deny LEM is to deny LNC.\footnote{By `deny' here I mean `assert the negation of', and, in turn, that to deny LEM is to assert something equivalent to the denial on LNC.} This is just an instance of DeMorgan. Note, many standard gappy logics, including Kleene, maintain the above inference. The aim, then, is to develop a logic that does two things. First, it blocks this inference, and second, it allows for failures of LEM. 

When discussing our general idea of a triangle\textemdash i.e. our idea of a triangle \em in general\em, not specifically a right triangle, equilateral triangle, scalene, or other specific kind of triangle\textemdash Locke writes:

\begin{quote}
... Does it not require some pains and skill to form the \em general Idea\em\ of a \em Triangle\em, (which is yet none of the most abstract, comprehensive, and difficult,) for it must be neither Oblique, nor Rectangle, neither Equilateral, Equicrural, nor Scalenon; but all and none of these at once. (IV.VII.9 p. 596)
\end{quote}
Locke does make the inference from our general idea of a triangle being none of these, to it being all of them. But why make this inference? Barnes tries to show that sense-data are deficient in a certain way\textemdash they are gappy. Why think because of this that they are deficient in some other way\textemdash e.g. contradictory? Now, there are \em other\em\ phenomenon that might lead one to believe that sense-data are contradictory (e.g. waterfall illusion). But these, I think, are different concerns, and depending on one's specific theory of sense-data, one could have explanations for these contradictory phenomena, independent of accepting the `gappy' phenomena. My point is that we should not let Barnes's argument show that sense-data are contradictory.

\subsection{Resulting Motivation}

The resulting picture is as follows. Let us suppose that our language is, by and large, classical. That is, our language is, for the most part, modeled correctly by classical logic. Many of the inferences we think are valid are the classical inferences. In fact, our language might be completely classical, if it were not for certain misbehaved objects. That is, let us further suppose that there are these ill-manored objects that in some sense `disobey' the LEM. As a result certain atomic sentences about those objects\textemdash and likewise their negations\textemdash are neither true nor false.

But it is nothing about our \em language\em\ per se that makes this the case. That is, we have, suppose, no reason to think that every sentence  whatsoever could be gappy. It is just these deficient `gappy' objects. Other than that, as before, our language is `classical'. In other words, let us suppose that we do not think that we have some special semantic predicate or funny sort of negation\textemdash both of which would be features of an essentially non-classical \em language\em\textemdash that is designed specifically to handle these cases. The next chapter develops the logic.

%\bibliographystyle{chicago}
%\bibliography{../Diss}
%\end{document}


